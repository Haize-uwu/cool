% Created 2024-08-06 Tue 12:52
% Intended LaTeX compiler: pdflatex
\documentclass[11pt]{article}
\usepackage[utf8]{inputenc}
\usepackage[T1]{fontenc}
\usepackage{graphicx}
\usepackage{longtable}
\usepackage{wrapfig}
\usepackage{rotating}
\usepackage[normalem]{ulem}
\usepackage{amsmath}
\usepackage{amssymb}
\usepackage{capt-of}
\usepackage{hyperref}
\usepackage{amsmath}
\usepackage{amssymb}
\usepackage{bussproofs}
\author{haize}
\date{\today}
\title{1logic}
\hypersetup{
 pdfauthor={haize},
 pdftitle={1logic},
 pdfkeywords={},
 pdfsubject={},
 pdfcreator={Emacs 29.4 (Org mode 9.8)}, 
 pdflang={English}}
\usepackage{biblatex}

\begin{document}

\maketitle
\tableofcontents

\section{Propositional Logic}
\label{sec:orgcdade8e}
\(A \implies B\)
This evalutes true if A is \uline{false} \textbf{OR} B is \uline{true}
\subsection{Logical Constants}
\label{sec:org7a05df5}
\(\top\): True propostion

\(\bot\): False propstion
\subsection{Variables}
\label{sec:orgbaa1885}
\begin{itemize}
\item p, q, r : lower case roman letters.
\end{itemize}
\subsection{Construction Trees}
\label{sec:org0028fa3}
\begin{itemize}
\item Taking any subtree is a valid logical formulae itself
\item the size of the graph is the number of nodes (variables + connectives)
\item The main connective is the one applied last, non-leaf node
\item create the tree from the inside
\end{itemize}
\subsection{Parsing Trees}
\label{sec:orgcc04d8b}
\begin{itemize}
\item Start constructing these from the main connective
\item Top to bottom
\end{itemize}
\subsection{Truth Tables}
\label{sec:orgf213b60}
\begin{itemize}
\item ternary function mapping \(\{0,1\} \rightarrow \{0,1\}\)
\end{itemize}
\subsection{Tautologies}
\label{sec:org8a0a8f8}
\begin{itemize}
\item propositionnaly valid formula that is always True.
\end{itemize}
\subsubsection{Notation}
\label{sec:org78f02d2}
\begin{itemize}
\item \(\models A\)
\end{itemize}
\begin{enumerate}
\item Examples
\label{sec:orgc2263e2}
\begin{itemize}
\item \(\models\) p \(\lor\) \textlnot{} p - Law of excluded middle
\item \(\models\) \textlnot{}(p \(\land\) \textlnot{} p) - Law of non-contradiction
\item \(\models\) ((p \(\land\)(p \(\implies\) q))\(\implies\) q) - modus ponents
\begin{itemize}
\item Set q false, left must be true so all true.
\end{itemize}
\end{itemize}
\end{enumerate}
\subsection{Contradictions}
\label{sec:orgfc8936a}
\begin{itemize}
\item always false formula
\end{itemize}
\begin{enumerate}
\item If a formula is not a contradiction, then it is satisfiable
\label{sec:orgc713418}
\end{enumerate}
\subsection{Propositional Logical Consequence}
\label{sec:org0236754}
$$A_1 , A_2 ,..., A_n \models C$$
\begin{itemize}
\item if C is true whenever all \(A_i\) are true.
\item \textbf{Tautology} is a special case of consequence
\end{itemize}
$$ \phi \models C = \models C$$
\subsubsection{Consiquence is reducible to validity}
\label{sec:org2fde082}
$$A_1 , ... , A_n \models B $$ is equivalent to: \(A_1 \land ... \land A_n \models B\)  \\
then $$\models (A_1 \land ... \land A_n) \implies B$$
\subsection{Sound Rules of propositional inference}
\label{sec:orgbafd668}
\begin{itemize}
\item A rule of propositional inference is a scheme where \(P_i\) are the premises we assume and \(C\) is the conclusion $$\frac{P_1 , ..., P_n}{C}$$
\item An inference rule is \textbf{logically sound} if $$P_1 , ..., P_n \models C$$
\item Putting concrete propostions (substituting for variables) is a propositional inference.
\end{itemize}
\subsection{How to check correctness}
\label{sec:orgfcb9638}
Check if a rule is sound : $$\frac{S, S \implies h}{h}$$ so check if \(S, S \implies h \overset{?}{\models} h\) \\
\(\therefore\) the inference is correct based on a sound rule, but can't say rules is sound
\subsection{Fallacies of Implication (Derivative Implication)}
\label{sec:orge2d72cb}
\begin{itemize}
\item converse: \(B \implies A\)
\item inverse: \textlnot{} A \(\implies\) \textlnot{} B
\item contrapositve: \textlnot{} B \(\implies\) \textlnot{} A this is \textbf{sound}.
\end{itemize}
\subsection{Truth assignments in propostional logical}
\label{sec:orgcffe910}
\begin{itemize}
\item PVAR : propostional variables (countably infinite)
\item FOR : set of all propositional formulas
\end{itemize}

A truth assignment is  a map $\backslash$\ v : PVAR \(\rightarrow\) \{T, F\} and \\
\(\bar{v}\):FORM \(\rightarrow\)\{T,F\}
\begin{enumerate}
\item Note the following:
\label{sec:orgd1dffb9}
\begin{itemize}
\item \(\bar{v}(\top)\) = T
\item \(\bar{v}(\bot)\) = F
\item \(\bar{v}(p)\) = v(p)
\end{itemize}
* $\bar{v}$(\not A) = $$\case{T , $\bar{v}$(A) = F}$$
\end{enumerate}
\subsection{Logical Equivalence}
\label{sec:org78ae8dd}
A \(\equiv\) B both obtain the same truth value under all truth valuations \\
$$A \equiv B \iff \bar{v}(A) = \bar{v}(B)$$ for all truth assignments if v
\subsubsection{De' Morgans Law}
\label{sec:org4652064}
\textlnot{}(p\(\land\) q) \(\equiv\) \textlnot{} p \(\lor\) \textlnot{} q
\subsubsection{this one}
\label{sec:orgab86df3}
p \(\land\) (p \(\lor\) q) \(\equiv\) p \(\land\) p \(\equiv\) p$\backslash$
\subsection{Relating Logical Equivalence, consequence and validity}
\label{sec:org01c9f93}
\begin{enumerate}
\item Reducible to validity
\label{sec:orgdb38a9f}
$$A \equiv B\  iff \models A \iff B $$
\begin{enumerate}
\item Proof in notes
\label{sec:orgbd934f0}
Suppose A \(\equiv\) B \\
Let v:PVAR\(\rightarrow\) \{T,F\} be arbitrary \\
since A \(\equiv\) B, \(\bar{v}\)(A) = \(\bar{v}\)(B) \\
\(\therefore\) \(\bar{v}\)(A \(\iff\) B) = T \\
Since v was arbitrary we conclude \\
\(\models\) A \(\iff\) B
\end{enumerate}
\item Reducible to Logical Consequence
\label{sec:org9eeb075}
A \(\equiv\) B iff both A \(\models\) B and \(B \models A\)
\end{enumerate}
\subsection{\(\equiv\) is an equivalence relation}
\label{sec:orgf62762c}
\begin{enumerate}
\item reflexive A\(\equiv\) A
\item symmetric if A\(\equiv\) B then B \(\equiv\) A
\item transitive if A \(\equiv\) B and B \(\equiv\) C then A \(\equiv\) C
\end{enumerate}
\subsection{\(\equiv\) is a congruence wrt to logical connectives}
\label{sec:org76c6592}
\begin{itemize}
\item if A \(\equiv\) B then \textlnot{} A \(\equiv\) \textlnot{} B \\
\item if \(A_1 \equiv B_1\) and \(A_2 \equiv B_2\) then (A\textsubscript{1} \(\cdot\) A\textsubscript{2} ) \(\equiv\) (B\textsubscript{1} \(\cdot\) B\textsubscript{2}) for all \(\cdot\) = \{\(\lor\), \(\land\), \(\implies\), \(\iff\)\}
\end{itemize}
\begin{enumerate}
\item Theorem Equivalent replacement
\label{sec:org85e75f6}
if A \(\equiv\) B then C(A/p) \(\equiv\) C(B/p) 

We replace p with A all at once, the same for B
\begin{enumerate}
\item Proof in notes \(\rightarrow\) by induction on C
\label{sec:orgff33bc4}
By Induction on C: 
\begin{enumerate}
\item \(C\) is \(\bot\) Then  \(C(A/p)\) = \(\bot\) and C(B/p) = \(\bot\) \\
\end{enumerate}
\(\therefore\)  \(C(A/p) = \bot \equiv \bot = C(B/p)\)
\begin{enumerate}
\item \(C\) is \(\top\) same as above.
\item 1. \(C\)  is \(q\) where \(q \neq  p\), then \(C(A/p) = q\) and \(C(B/p) = q\) \\
\end{enumerate}
SO \(C(A/p) = q \equiv q = C(B/p)\)
\begin{enumerate}
\item 2. \(C\) is \(p\) , then \\
\end{enumerate}
\(C(A/p) = A \equiv B = C(B/p)\) \\
Inductive Hypothesis:  Suppose holds for formulas \(C_1\) and \(C_2\) \\
\(C_1 (A/p) \equiv C_1 (B/p)\) and \\
\(C_2(A/p) \equiv C_2(B/p)\) 
\begin{enumerate}
\item C  is \textlnot{} C : \\
\end{enumerate}
Then \(C(A/p) = \not C(A/p) \overset{\star}{\equiv} \not C(B/p) =C(B/p)\), \\
Where \(\star\) holds by the Inductive Hypothesis and the fact that \(\equiv\) is a congruence wrt to propositional connectives. 
\begin{enumerate}
\item \(C\) is \((C_1 \cdot C_2)\) for some \(\cdot = \{\land \lor \implies \iff \}\) \\
\end{enumerate}
Then \(C(A/p) = (C_1(A/p) \cdot C_2(A/p))\) but by inductive hypothesis \\
\(C_1 (A/p) \equiv C_1 (B/p)\) and \\
\(C_2(A/p) \equiv C_2(B/p)\)  \\
So by the fact that \(\equiv\) us a congruence wrt to propositional connectives; \\
\(C(A/p) = ((C_1(A/p) \cdot C_2(A/p))\equiv(C(A/p) = (C_1(B/p) \cdot C_2(B/p))=C(B/p)\) \\
\(\therefore\) C(A/p) \(\equiv\) C(B/p)
\end{enumerate}
\end{enumerate}
\subsubsection{Important Logical Equivalence}
\label{sec:orgacba5bb}
\begin{itemize}
\item Idempotency : applying more than once makes no difference p\(\land\) p and p\(\lor\) p
\item Commutativity:
\item Associativity:
\item Distributivity:
\item BUNCH OF STUFF AT THE END OF SECTION 1.3
\end{itemize}
\begin{enumerate}
\item Know important \(\equiv\) for negation how it affects \(\implies\) and \(\iff\)
\label{sec:org2da9448}
\end{enumerate}
\subsection{Inductive Definitions}
\label{sec:org564c100}
\subsubsection{Structured induction}
\label{sec:org0eccdc6}
Important when an infinite set of structured objects are to be defined 

BNF \\
$$w:= \epsilon | wa$$,  where \(a\in A\)
\begin{enumerate}
\item Example proof The number of right parenthesis is the same as left in logical formalae\hfill{}\textsc{ATTACH}
\label{sec:orgd75f034}


\emph{l} : \(\mathbf{FOR} \rightarrow \mathbb{N}\) \\
\emph{r} : \(\mathbf{FOR} \rightarrow \mathbb{N}\) \\
To prove /l/(\(\psi\))= /r/(\(\psi\)) \(\forall \psi\) \(\in\) \(\mathbf{FORM}\)


\begin{enumerate}
\item if \(\psi\) is \(\top\), then \emph{l} (\(\psi\)) = \emph{l} (\(\top\)) = 0 = \emph{r} (\(\top\)) = \emph{r} (\(\psi\)) \\
if \(\psi\) is \(\bot\), then \emph{l} (\(\psi\)) = \emph{l} (\(\bot\)) = 0 = \emph{r} (\(\bot\)) = \emph{r} (\(\psi\))
\item if \(\psi\) is \(p\) then \emph{l} (\(\psi\)) = \emph{l} (p) = 0 = \emph{r} (p) = \emph{r} ($\backslash$\(\psi\))
\item Suppose \emph{l} (A) = \emph{r} (A) and A is \textlnot{} A: \\
\end{enumerate}
then \emph{l} (\(\psi\)) = \emph{l} (\textlnot{} A) = \emph{l} (A) =/r/ (A) = \emph{r} (\textlnot{} A) = \emph{r} (\(\psi\))
\begin{enumerate}
\item Suppose \emph{l/(A)  = /r} (A) and \emph{l} (B) = \emph{r} (B) and \(\psi\) is (A \(\cdot\) B), for \(\cdot\) = \{\(\land\), \(\lor\), \(\implies\), \(\iff\)\} \\
\end{enumerate}
Then \emph{l} (\(\psi\)) = \emph{l} ((A \(\cdot\) B) = \emph{l} (A) + \emph{l} (B) + 1 \\
Now by IH = \emph{r} (A) + \emph{r} (B) +1 = \emph{r} ((A \(\cdot\) B)) = \emph{r} (\(\psi\))
\end{enumerate}
\subsection{Complex Proof}
\label{sec:org53607b5}
We will prove \(A \land B \implies B \implies A\) using natural deduction.

\begin{LATEX}
\begin{prooftree}
  \AxiomC{$A \land B$}
  \UnaryInfC{$A$}
  \AxiomC{$A \land B$}
  \UnaryInfC{$B$}
  \BinaryInfC{$B \land A$}
\end{prooftree}
\end{LATEX}
\end{document}
