% Created 2024-08-06 Tue 17:52
% Intended LaTeX compiler: pdflatex
\documentclass[11pt]{article}
\usepackage[utf8]{inputenc}
\usepackage[T1]{fontenc}
\usepackage{graphicx}
\usepackage{longtable}
\usepackage{wrapfig}
\usepackage{rotating}
\usepackage[normalem]{ulem}
\usepackage{amsmath}
\usepackage{amssymb}
\usepackage{capt-of}
\usepackage{hyperref}
\usepackage{amsmath}
\usepackage{amssymb}
\usepackage{bussproofs}
\author{haize}
\date{\today}
\title{Deductive Reasoning in Propositional Logic}
\hypersetup{
 pdfauthor={haize},
 pdftitle={Deductive Reasoning in Propositional Logic},
 pdfkeywords={},
 pdfsubject={},
 pdfcreator={Emacs 29.4 (Org mode 9.8)}, 
 pdflang={English}}
\usepackage{biblatex}

\begin{document}

\maketitle
\tableofcontents

\section{Deductive Systems}
\label{sec:org0b516f7}

\begin{enumerate}
\item Logical Consequence:
\label{sec:org9c5f188}
\[ A_1 , ..., A_n \models C
\]
\begin{itemize}
\item and Logical validity:
\end{itemize}
\[ \models C\]
These are semantic notions \(\rightarrow\) meaning of formulae. \\
They require checking truth in all models for logic.
\item Deductive systems:
\label{sec:orge606e53}
\begin{itemize}
\item Are meant to capture the logical consequence and validity defined by the logical semantics, in terms of logical \textbf{\textbf{logical deductions (derivations)}}.
\item A Logical derivation is a mechanical procedure, not reffering the meaning of the occuring formulae.
\item In deductive systems logical consequence is replaced by \textbf{\textbf{deductive consequence}} and valid formulae (tautologies) - by derivable formulae (theorems).
\end{itemize}
\end{enumerate}
\subsection{Basic Components of a deductive System}
\label{sec:org6134568}
\begin{itemize}
\item Formal Logical language
\item Axioms (Rules of inference)
\item Inference (deduction) from a set of assumptions in a given deductive System \(D\):
\end{itemize}
\[A_1, ... A_n \vdash_D C\]
\begin{itemize}
\item Formulae derivable from no assumptions are called \textbf{\textbf{Theorems}} of \(D\):
\end{itemize}
\[\vdash_D C\]
\subsection{Soundness and completeness of a deductive system}
\label{sec:orge2f9323}
\begin{itemize}
\item A deductive system \(D\) is \textbf{sound} (correct) for a given logical semantics if \(D\) can \textbf{\textbf{only}} derive what is logically correct (valid):
\end{itemize}
\[A_1,...A_n \vdashD C \implies A_1, ... A_n \models C\]
\begin{itemize}
\item In particular:
\end{itemize}
\[\vdash_D C \implies \models C\]

\begin{itemize}
\item A system \(D\) is \textbf{Complete} for a given logical semantics if \(D\) can derive \textbf{every} valid logical consequence:
\end{itemize}
\[ A_1, ... A_n \models C \implies\ A_1,...A_n \vdashD C\]
\begin{itemize}
\item In particular
\end{itemize}
\[\models C \implies \vdash_D C \]
\begin{itemize}
\item A deductive system is *adequte for a given semantics if it is both sounnd AND complete:
\end{itemize}
\[ A_1, ... A_n \models C \iff \ A_1,...A_n \vdashD C\]
\begin{itemize}
\item in particular \[\models C \iff \vdash_D C \]
\end{itemize}
\subsection{Propositional natural deduction}
\label{sec:org73ef3b8}
\begin{itemize}
\item Natural Deduction (ND): System for structural logic derivation from a set of assumptions, based on rules, specific to the logical connectives.
\item There are Introduction and elimination rules for each connective.
\item Intro and cancelation of assumptions, assumptions can be reused many times  before being discharged.
\item Cancelation of assumptions: only when a rule allows it, not an obligation.
\item All assumptions left at the end of a derivation must be declared.
\end{itemize}
\subsubsection{NOTE:}
\label{sec:org8f2b56e}
The fewer assumptions, the stronger the claim of the derivation.
\subsubsection{Rules for propositional connectives}
\label{sec:org6db2ecf}

\begin{itemize}
\item Refer to notes!
\end{itemize}
\begin{enumerate}
\item Examples in notes!
\label{sec:orge89eb98}
\end{enumerate}
\item Strategies for ND proofs in propositional logic
\label{sec:org8843109}
Given a set of premises \(\Delta\) and the goal of trying to prove \(\Gamma\):
\begin{enumerate}
\item Apply premises \(p_i \in \Delta\) to prove \(\Gamma\)
\item If you need to use a \(\lor\)-premise, then apply \(\lor\)-elimination to prove \(\Gamma\) for each disjunct
\item Otherwise work backwards from the goal:
\begin{enumerate}
\item If the goal \(\Gamma\) is a conditional \((A \implies B)\) then assume A and prove B using \(\rightarrow\)-introduction.
\item IF the goal \(\Gamma\) is negative (\textlnot{} A), then assume (\textlnot{} \textlnot{} A) and prove contradiction; use the \textlnot{}-introduction.
\item if the goal \(\Gamma\) is a conjunction (A \(\land\) B) then prove A and prove B; use \(\land\)-introduction.
\item If the goal \(\Gamma\) is a disjunction (A \(\lor\) B) then prove one of A or B; use \(\lor\)-introduction.
\end{enumerate}
\end{enumerate}
\begin{enumerate}
\item Examples from the oxford pack - click on them to go to solutions!
\label{sec:orge6e0b88}

\begin{itemize}
\item \href{http://logicmanual.philosophy.ox.ac.uk/pdfslides/p1.pdf}{sol}
\end{itemize}
\begin{center}
\includegraphics[width=.9\linewidth]{../images/20240806-174005_screenshot.png}
\end{center}
\begin{itemize}
\item \href{http://logicmanual.philosophy.ox.ac.uk/pdfslides/p2.pdf}{sol} 
\begin{center}
\includegraphics[width=.9\linewidth]{../images/20240806-174111_screenshot.png}
\end{center}
\item \href{http://logicmanual.philosophy.ox.ac.uk/pdfslides/p3.pdf}{sol}
\end{itemize}
\begin{center}
\includegraphics[width=.9\linewidth]{../images/20240806-174208_p31.jpg}
\end{center}
\begin{itemize}
\item \href{http://logicmanual.philosophy.ox.ac.uk/pdfslides/p4.pdf}{sol}
\begin{center}
\includegraphics[width=.9\linewidth]{../images/20240806-174334_p41.jpg}
\end{center}
\item \href{http://logicmanual.philosophy.ox.ac.uk/pdfslides/p5.pdf}{sol}
\end{itemize}

\begin{center}
\includegraphics[width=.9\linewidth]{../images/20240806-174426_p51.jpg}
\end{center}
\end{enumerate}
\end{enumerate}
\section{Useful Resources for Natural deductions}
\label{sec:org3c83658}
\begin{itemize}
\item \href{https://users.ox.ac.uk/\~logicman/}{Oxford logic pack}
\end{itemize}
\end{document}
